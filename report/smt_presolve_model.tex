\documentclass{article}
\usepackage{amsmath}
\usepackage{amssymb}
\usepackage{algorithm}
\usepackage{algorithmic}
\usepackage{listings}
\usepackage{xcolor}

\title{SMT Model with Presolving for Sports Tournament Scheduling}
\author{CDMO Project 2024/2025}
\date{\today}

\begin{document}

\maketitle

\section{Overview}

This document describes our SMT (Satisfiability Modulo Theories) approach for solving the Sports Tournament Scheduling (STS) problem, which employs a sophisticated presolving technique followed by an efficient period assignment encoding.

\section{Graph-Theoretic Foundation}

The STS problem has a deep connection to graph theory:

\subsection{Complete Graph Representation}
The tournament can be modeled as a \textbf{complete graph} $K_n$ where:
\begin{itemize}
    \item Vertices represent teams
    \item Edges represent matches between teams
    \item Each edge must be scheduled exactly once
\end{itemize}

\subsection{Edge Coloring and Decomposition}
The scheduling problem is equivalent to:
\begin{enumerate}
    \item \textbf{1-factorization} of $K_n$: Decompose $K_n$ into $n-1$ perfect matchings (weeks)
    \item \textbf{Period assignment}: Assign each edge in each perfect matching to a time slot (period)
\end{enumerate}

A \textbf{perfect matching} is a set of edges where each vertex appears exactly once. For $K_n$ with even $n$, each perfect matching contains exactly $\frac{n}{2}$ edges.

\subsection{Theoretical Properties}
\begin{itemize}
    \item $K_n$ has exactly $\binom{n}{2} = \frac{n(n-1)}{2}$ edges (total matches)
    \item $K_n$ can be decomposed into exactly $n-1$ perfect matchings (Baranyai's theorem)
    \item Each perfect matching represents one week of the tournament
    \item The circle method generates one specific 1-factorization of $K_n$
\end{itemize}

\section{Problem Decomposition}

The key insight of our approach is to decompose the STS problem into two phases:

\begin{enumerate}
    \item \textbf{Presolving Phase}: Generate a complete round-robin schedule using the circle method
    \item \textbf{SMT Phase}: Assign periods to the pre-determined matches
\end{enumerate}

This decomposition dramatically reduces the search space from $O(n!)$ possible schedules to a much smaller period assignment problem.

\section{Phase 1: Circle Method Presolving}

\subsection{Round-Robin Generation}

The circle method is a classical algorithm for generating round-robin tournaments that constructs a 1-factorization of $K_n$. For $n$ teams, we designate team $n$ as a \textit{pivot} and arrange the remaining teams $\{1, 2, \ldots, n-1\}$ in a circle.

\textbf{Graph-theoretic interpretation}: At each round (week), the pivot plays against one team on the circle, while the remaining teams are paired by taking symmetric positions around the circle. This ensures each vertex (team) appears in exactly one edge (match) per perfect matching (week).

\begin{algorithm}
\caption{Circle Method for Round-Robin Generation}
\begin{algorithmic}[1]
\REQUIRE $n$ teams (even number)
\STATE $pivot \leftarrow n$
\STATE $circle \leftarrow [1, 2, \ldots, n-1]$
\STATE $weeks \leftarrow n - 1$
\FOR{$w = 0$ \TO $weeks - 1$}
    \STATE $matches[w] \leftarrow [(pivot, circle[w])]$
    \FOR{$k = 1$ \TO $\lfloor n/2 \rfloor - 1$}
        \STATE $i \leftarrow circle[(w + k) \bmod (n-1)]$
        \STATE $j \leftarrow circle[(w - k) \bmod (n-1)]$
        \STATE $matches[w] \leftarrow matches[w] \cup \{(i, j)\}$
    \ENDFOR
\ENDFOR
\end{algorithmic}
\end{algorithm}

This algorithm ensures that:
\begin{itemize}
    \item Every team plays every other team exactly once
    \item Every team plays exactly once per week
    \item The generated schedule is always feasible
\end{itemize}

\subsection{Home/Away Balance Heuristic}

After generating the round-robin pairings, we apply a heuristic to determine home/away assignments that promote balance:

For each match $(i, j)$:
\begin{equation}
d = (j - i) \bmod n
\end{equation}

\begin{equation}
\text{Home team} = \begin{cases}
i & \text{if } d < \frac{n}{2} \\
j & \text{otherwise}
\end{cases}
\end{equation}

This heuristic tends to balance the number of home and away games across teams.

\section{Phase 2: SMT Period Assignment}

\subsection{Decision Variables}

Given the precomputed matches $M_w = \{(i_1, j_1), (i_2, j_2), \ldots\}$ for each week $w$, we define:

\begin{equation}
p_{w,i,j} \in \{1, 2, \ldots, \frac{n}{2}\} \quad \forall w, \forall (i,j) \in M_w
\end{equation}

where $p_{w,i,j}$ represents the period assigned to match $(i,j)$ in week $w$.

\subsection{Constraints}

\subsubsection{Constraint 1: One Match Per Period Per Week}
For each week $w$ and period $k$:
\begin{equation}
\sum_{(i,j) \in M_w} [p_{w,i,j} = k] = 1
\end{equation}

\subsubsection{Constraint 2: Team Period Limit}
Each team $t$ can play at most twice in the same period across all weeks:
\begin{equation}
\sum_{w} \sum_{(i,j) \in M_w : t \in \{i,j\}} [p_{w,i,j} = k] \leq 2 \quad \forall t, \forall k
\end{equation}

\subsection{Symmetry Breaking}

To reduce the search space, we fix the period assignments for the first week:
\begin{equation}
p_{w_1, i_j, j_j} = j \quad \forall j \in \{1, 2, \ldots, \frac{n}{2}\}
\end{equation}

where $(i_j, j_j)$ are the matches in the first week sorted lexicographically.

\section{SMT Encoding Details}

\subsection{Pseudo-Boolean Constraints}
We use pseudo-Boolean constraints instead of arithmetic constraints for efficiency:
\begin{itemize}
    \item \texttt{PbEq(guards, 1)} for exactly-one constraints
    \item \texttt{PbLe(guards, 2)} for at-most constraints
\end{itemize}

\subsection{Solver Optimization}
We employ Z3's \texttt{card2bv} tactic to convert cardinality constraints to bit-vector operations:
\begin{equation}
\text{solver} = \text{Then}(\text{'card2bv'}, \text{'smt'}).\text{solver}()
\end{equation}

\section{Algorithm Complexity}

\begin{itemize}
    \item \textbf{Presolving}: $O(n^2)$ time to generate round-robin and apply balance heuristic
    \item \textbf{Variables}: $O(n^2)$ period assignment variables (vs. $O(n^3)$ in direct encoding)
    \item \textbf{Constraints}: $O(n^2)$ constraints (vs. $O(n^3)$ in direct encoding)
\end{itemize}

\section{Performance Analysis}

The presolving approach achieves dramatic performance improvements:

\begin{center}
\begin{tabular}{|c|c|c|}
\hline
$n$ & Direct SMT & Presolve SMT \\
\hline
12 & 30s & $<$1s \\
18 & timeout & 1s \\
20 & timeout & $<$100s \\
\hline
\end{tabular}
\end{center}

The efficiency gains come from:
\begin{enumerate}
    \item \textbf{Reduced search space}: No need to search for valid round-robin pairings
    \item \textbf{Fewer variables}: Only period assignments, not match scheduling
    \item \textbf{Better constraint structure}: Pseudo-Boolean constraints are more efficient
    \item \textbf{Solver optimization}: Card2bv tactic optimizes cardinality handling
\end{enumerate}

\section{Implementation Notes}

The implementation is structured in three key functions:
\begin{itemize}
    \item \texttt{circle\_matchings(n)}: Generates round-robin using circle method
    \item \texttt{home\_away\_balance(matches, n)}: Applies balance heuristic
    \item \texttt{z3\_label\_periods(matches, periods)}: Solves period assignment via SMT
\end{itemize}

This modular design allows for easy experimentation with different presolving strategies and balance heuristics.

\end{document}